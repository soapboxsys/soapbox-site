\documentclass[a4paper]{article}

\usepackage[english]{babel}
\usepackage[utf8x]{inputenc}
\usepackage{amsmath}
\usepackage{graphicx}
\usepackage[colorinlistoftodos]{todonotes}


\title{Prototype Fund Final Report}
\author{
{\rm Alex Kuck}\\
Soapbox Systems\\
alex@soapbox.systems
\and
{\rm Nick Skelsey}\\
Soapbox Systems\\
nick@soapbox.systems
\and
{\rm Alex Evans}\\
Soapbox Systems\\
evans@soapbox.systems
}

\begin{document}
\maketitle

\section{Purpose and Introduction} \index{Purpose}

The purpose of this report is to provide an overview of the activities conducted by Soapbox Systems under the Knight Prototype Fund on Ombuds. The document presents key insights and reflections from the development of the Ombuds prototype and various conversations that Soapbox Systems has conducted with relevant stakeholders.\par
Under the Knight Grant, the team at Soapbox Systems set out to evaluate the hypothesis: \textit{Can a text only peer-to-peer immutable microblog address the problems of speech disappearing from the web?} In retrospect, the intertwined problems of permanence, censorship, and content availability are not easily addressed by a hypothesis as broad as this.

\section{Outreach Efforts}
The conversations held about Ombuds and its applications to enhance free expression on the web and preserve user-generated content can broadly be divided into three categories: 1) Nonprofits at the intersection of technology and policy, 2) digital and international rights focused nonprofits, and 3) individual journalists and activists in the space.\par

\subsection{Technology and Policy Nonprofits}
The Soapbox Systems team met with the chief technologist and members of the free expression group of the Center for Democracy and Technology (CDT) at their offices in Washington, D.C. The CDT team reacted positively to our technology, posed insightful questions, and recommended points of contact at organizations in the bitcoin space.\par
The greater Washington, D.C. area proved to have many organizations and events focused on the intersection of these two fields. It is worth noting that while there is a dirth of organizations advising governmental agencies their are few it appears in the human rights space that are actually authoring novel technology. Some notable exceptions that we interacted with include the Open Tech Institute, HacDC, and the Cybersecurity Center at UMD.

\subsection{Digital and International Rights Nonprofits}
Organizations that are active in assisting digital activists and human rights advocacy, such as Amnesty International USA and Human Rights Watch also provided valuable feedback on the project. Our conversations with representatives at these groups indicated that if Ombuds were shown to be fully functional and its value had been sufficiently demonstrated and tested in the field, then these organizations would consider using the tool. An essential roadblock is that these organizations are frequently hesitant to demo and test the technology as they frequently operate in high-risk environments.\par

For example, Human Rights Watch (HWR) has more than 70 researchers placed in over 50 different nations. These researchers handle mostly communication with HRWs main offices and their sources on the ground who are frequently at risk for their mere affiliation with HRW. This tightly confined circle of operation is becoming more tenuous as some nation states are realizing that they can actively curtail NGOs operating within their civil society with little international repercussions.\par

Tactical Tech, an organization that works directly with activists and provides technical resources and advice to safely pursue their goals, provided in-depth technical advice for the development of the project as well.\par

\subsection{Journalists and Activists}
We also reached out to several journalists that cover digital activism and work with digital activists, with varied responses. Often, these groups expressed skepticism about deploying a tool like Ombuds in high-risk censorship-prone countries. Indicatively, the head of Global Voices Advocacy was quick to advise extreme caution when deploying our software. As that community regularly sees the fallout of internet freedom projects, it was repeatedly emphasized to us that extensive research and direct communication with at-risk users is essential to guarding against flaws that could make such tools counterproductive or even dangerous for their intended audiences. \par

On the journalism side, organizations that work with user-generated content such as Eyewitness Media Hub and the First Draft Coalition were also very receptive to the underlying technology of Ombuds. One of the major problems that these organizations are seeking to address is the issue of disappearing eyewitness content and the need for a reliable permanent record for content generated on the web, especially in light of verifying stories published at earlier times (such as the Arab Spring uprisings) and providing evidence in international court proceedings. A member of the Tow Center validated the need to protect such content as it is generated on the web and was welcoming of potential contributions from Soapbox Systems.\par

\section{Funding Environment}

Ultimately, the six-month time frame proved insufficient to fully evaluate the aforementioned hypothesis. The full determination of the potential usefulness of an immutable text-only microblogging platform, as well related uses of Ombuds, would require additional software development, testing, and partnership with organizations in the space for deployment.\par

Open-source technology is typically not attractive to equity investors and the grant-making process allows little hope for recurring guaranteed awards. As such, projects such as Ombuds frequently fail to acquire subsequent funding after completion of a proof-of-concept or functional prototype as the requested amounts for deployment and further development are often substantially higher.\par

There are a number of organizations that are active in the space (NED, Media Democracy Fund, OTF, Sloan, Ford, USAID, among others), but grant seekers must rigorously demonstrate the pain points that their proposed projects address. Moreover, the activities of proof-of-concept or prototype development grants typically focus heavily on the development side and do not involve securing additional sources of support or longer-term funding.\par

\subsection{Limited Use}

A critical mistake that we made early on was that we incorrectly characterized Ombuds as censorship resistant technology. The tool is really an end point once a user has a route around censorship. The problems it can address are thus more limited than originally ? albeit still important. We feel that we appropriately responded to this specific critique and changed our language to address it. \par

\section{Identified Areas of Growth}

Public record-keeping and archiving of user generated content (UGC) pose significant challenges to new media organizations and journalists in the space. The members of the First Draft Coalition indicated that the problem of user content disappearing from the Web is one of the major current areas of focus for its member organizations. \par

At the moment, former employees at organizations such as Storyful who are involved in the creation, discovery, verification, and publication of UGC are noting that media organizations should be also become active in the preservation of such content. Given that UGC can be essential to historical archiving and evidence-gathering by third parties (e.g. international courts), the challenge for these organizations is to be able to recover removed content as well as to verify that it matches the original version.\par

The terms of service agreements of publishers (e.g. YouTube, Twitter) pose further complications as journalists are often prohibited from directly downloading UGC that they locate on the Web. Beyond the requisite technological infrastructure to support such archival work, case-by-case determinations of whether archiving constitutes fair use, respects the privacy of the publisher and the individual to whom the content pertains are essential to any UGC archive. Furthermore, a record that serves the function of future verification and is designated for purely non-commercial archival purposes is more likely to be protected under fair use.\par

\section{Conclusion}
	After many conversations, Soapbox Systems internally confirmed that Internet censorship is on the rise and that sensitive content posted to the Web can disappear from the Web permanently. It appears that governments worldwide have caught up with civil society in their technical understanding of the Internet and the Web. With the use of the Web growing in Africa, South Asia, and the Middle East the world is starting to learn about new challenges to digital free expression daily. It is no longer sufficient to point at the Great Firewall as the single example of how repressive nations can control their citizens? digital lives.\par	
	This means that new tools, new methods, and new ideas are needed to route around censorship and preserve social history before it is scrubbed clean by determined censors. While we do not feel that Ombuds lived up to its initial promise of a system that could combat censorship, we are proud to say that this sort of work is essential if we are to one day live in a free and just world.\par

\end{document}